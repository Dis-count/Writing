\documentclass[UTF8]{article}
    \author {Li Zikang}
    \title {Homework 1}
\usepackage{ctex}
\usepackage{amsmath}
\usepackage{amssymb}
\date{}
\begin{document}
    \maketitle
\section*{2-5} (a) Write donw a recursion for $f^{(n)} (x)$ in terms of $f^{(n-1)} (\cdot)$ and the parameters of the problem.
    \large

    For any participant, with $n$ more spins, he/she can choose to spin or stop. When choosing \textbf{stopping}, he/she will take away the  current accumulation of $x$. We can obtain $f^{(n)} (x)$ = $x$. While choosing \textbf{spinning}, it is also easy to know that

    $ f^{(n)}(x)= \sum_{i} f^{(n-1)}(x+r_i)\cdot p_i$, where $i = 1,\ldots,m.$\\

    $ f^{(n)}(x)= \max\{\sum_{i} f^{(n-1)}(x+r_i)\cdot p_i, ~x\}$

    \normalsize
    (b) Solve the problem numerically for the case in which $m=3, p_i = 1/4$, \text{for} $i = 1,2,3, r_1 =1, r_2 =5, r_3 =10$, and there are one or two more spins available.

    \large
    At the beginning, the winning is $0$, so must choose to spin.
    $$ f^{(2)}(0)=\left\{
    \begin{aligned}
    & f^{(1)}(1), \text{probability} ~1/4 \\
    & f^{(1)}(5), \text{probability} ~1/4 \\
    & f^{(1)}(10), \text{probability} ~1/4 \\
    & 0 ~\quad \text{probability} ~1/4
    \end{aligned}
    \right.
    $$

    $$ f^{(1)}(x)=\left\{
    \begin{aligned}
    & x+1, \text{probability} ~1/4 \\
    & x+5, \text{probability} ~1/4 \\
    & x+10, \text{probability} ~1/4 \\
    & 0 ~\quad \text{probability} ~1/4 \\
    & x ~\quad \text{choose to stop}
    \end{aligned}
    \right.
    $$

So when spinning once, the excepted winnings are $1/4 \times (1+5+10)=4$. And when spinning twice, $f^{(1)}(x) = 1/4 \times [(x+1)+(x+5)+(x+10)]$. $f^{(2)}(0) =1/4 \times (f^{(1)}(1)+f^{(1)}(5)+f^{(1)}(10))= 1/4 \times (4+3/4+4+15/4+4+30/4) = 6$

In particular, $f^{(0)}(x) = x, f^{(1)}(x) =\max\{3/4x+4, x\}.$
Let $3/4x+4 = x$, $x = 16$. When $x>16$, the participant will take away the winnings; when $x<16$, the participant will spin again. Consequently, the expected total winnings are 16 for this case.

\section*{2-6}

(a) Write down a recursion for $v(x)$ and use it to find an explicit expression for $v(x)$ in terms of $x, \Phi(x), I(x), \text{and}~ \alpha,$ where $I(x):=E(X-x)^{+}=\int_{x}^{\infty}(\xi-x) \phi(\xi) d \xi$. \\

$v(x,n) = v(x,n-1)\cdot \alpha \cdot \Phi(x) + \int_{x}^{\infty} \xi \phi(\xi) d \xi$ \\

Let $V =\int_{x}^{\infty}\xi \phi(\xi) d \xi.$ \\

$v(x,n) = v(x,1)\cdot [\alpha \cdot \Phi(x)]^{(n-1)} + V[1+[\alpha \cdot \Phi(x)]+[\alpha \cdot \Phi(x)]^2+\cdots+[\alpha \cdot \Phi(x)]^{n-2}]$ \\

$v(x,1) = \int_{x}^{\infty}\xi \phi(\xi) d \xi = V$

Then we can obtain that $v(x) = V\frac{1-(\alpha \cdot \Phi(x))^n}{1-\alpha \cdot \Phi(x)}$, $\alpha \in (0,1)$.
Because this is an infinite period, let $n \to \infty$, so $v(x) = \frac{V}{1-\alpha \cdot \Phi(x)}$.

$I(x) = \int_{x}^{\infty}(\xi-x) \phi(\xi) d \xi = \int_{x}^{\infty} \xi \phi(\xi) d \xi - x(1-\Phi(x))$ \\

So $V = I(x) + x(1-\Phi(x))$, substitute it in the $v(x)$, we get $v(x) = \frac{I(x) + x(1-\Phi(x))}{1-\alpha \cdot \Phi(x)}$ \\

(b) $\alpha = 0.99$ bids $\sim N(1000,200)$. x = 1200.
So $v(1200) =  \frac{I(1200)+ 1200(1-\Phi(1200))}{1-\alpha \Phi(1200)}$, $I(1200) = 200I_N (1)$, $\Phi(1200) = \Phi_N(1) = 0.841$,$I_N(1) = \phi_N(1) - 1+ \Phi_N(1) = 0.083$.\\
By calculating, $v(1200) = 1238.9$

\section*{2-7}
(a) Display $\Omega$.

$\Omega =\{(3,3,1),(2,3,2),(2,2,2),(1,3,2),(2,3,1),(0,3,2),(1,3,1),\\(1,1,2),(2,2,1),(2,0,2),(3,0,1),(1,0,2),(2,0,1),(1,1,1),(0,0,2)\}$

(b) $S_0 = \{(3,3,1)\}$, $S_1 = \{(2,3,2),(2,2,2),(1,3,2)\}$,\\
$S_2 = \{(2,3,1)\}$, $S_3 = \{(0,3,2)\}$,\\
$S_4 = \{(1,3,1)\}$, $S_5 = \{(1,1,2)\}$,\\
$S_6 = \{(2,2,1)\}$, $S_7 = \{(2,0,2)\}$,\\
$S_8 = \{(3,0,1)\}$, $S_9 = \{(1,0,2)\}$, \\
$S_{10} = \{(2,0,1),(1,1,1)\}$, $S_{11} = \{(0,0,2)\}$.

m = 11.

\section*{3-4}
(a) Write down a recursion for $f^n(x|S)$.

 $f^{(n)}(x|S)= \sum_{i} f^{(n-1)}(x+r_i|S)\cdot p_i$.

 $$ f^{(n)}(x|S)=\left\{
 \begin{aligned}
 & \sum_{i} f^{(n-1)}(x+r_i|S)\cdot p_i,~ \text{spin or}~ x<S.  \\
 & x ~\quad, \text{stop}.
 \end{aligned}
 \right.
 $$

 (b) Being indifferent means $\sum_{i} f^{(n-1)}(S+r_i|S)\cdot p_i = S = f^{(n)}(S|S) \Rightarrow \sum_i^m(S^*+r_i)p_i = S^* \Rightarrow S^* = \frac{\mu}{p}$.

\section*{3-5}
(a) Write a recursion for $f_n$ in terms of $f_{n-1}$

$f_n = f_{n-1} \cdot \alpha \cdot \Phi(\alpha f_{n-1})+ \int_{\alpha f_{n-1}}^{\infty}\xi \phi(\xi) d \xi$

Let $V = \int_{\alpha x}^{\infty}\xi \phi(\xi) d\xi = I(\alpha x) + \alpha x (1-\Phi(\alpha x))$.

(b) Calculate $f_n$.
$I_N(-0.05) = \phi_N(-0.05) -1 + \Phi_N(-0.05) = 0.398 -1 + 0.48 = -0.122$,

$V = I(990) + 990(1-\Phi(990)) = 200I_N(-0.05) + 990(1-\Phi_N(-0.05)) = 200\times(-0.12) + 990(1-0.48)= 490.5 $\\
$f_2 = \alpha\mu \Phi(\alpha \mu) + \int_{\alpha \mu}^{\infty}\xi \phi(\xi) d \xi = 0.48*0.99*1000 + 490.5 = 965.7$

$f_3 = \alpha \Phi(\alpha f_2) f_2 + \int_{\alpha f_2}^{\infty}\xi \phi(\xi) d \xi = 916.5$

$f_4 = \alpha \Phi(\alpha x) f_3 + \int_{\alpha x}^{\infty}\xi \phi(\xi) d \xi = 843.3$

(c) Let $f_n = f_{n-1}$, we can obtain that
$f_n = \alpha f_n \Phi(\alpha f_n) + I(\alpha f_n)+ \alpha f_n (1-\Phi(\alpha f_n))$.
Obviously, we can obtain that $f_n = \frac{I(\alpha f_n)}{1-\alpha}$.

\section*{3-6}

(a) Because there is a discount factor which is less than 1, when you wait it for a long period, your profit will be very low as time went by. So, it is obvious that we don't need to wait for all bids to be received.

(b)
$$ f_n(x)=\left\{
\begin{aligned}
& \alpha f_{n-1}(x/\alpha), \text{not accept this period}.  \\
& \alpha^t x ~\quad, \text{accept}.
\end{aligned}
\right.
$$

$f_n(x) = \alpha f_{n-1}(x/\alpha) \int_{x/\alpha}^{\infty} \phi(\xi) d \xi + \alpha^t x \Phi(x/\alpha)$ \\
To be honest, I'm confused by this question....

\section*{4-4}
The original formulation is
$v(x, S)=\left\{\begin{array}{ll}c & \text { if } x=0 \\ p x+q v(x-1, S) & \text { if } 1 \leq x \leq S \\ v(S, S) & \text { if } S<x\end{array}\right.$,
for this case, we can continue drive until we find an unoccupied space. So define a function $F(x),f(x,i)$, where $x<0$.

$F(x) = pf(x,0)+qf(x,1)$ ~and

$f(x, i)=\left\{\begin{array}{ll}
-x & \text { if } i=0 \\
F(x-1) & \text { if } i=1
\end{array}\right.$

$i= 0$ means the space isn't occupied, only choose to park here.
Now, we can obtain $F(x) = p(-x)+qF(x-1) = p(\sum_{i=-x}^{n-x-1} iq^{i+x}) = \frac{q[1+(n-1)q^n-nq^{n-1}]}{(1-q)^2}-x\frac{1-q^n}{1-q}
, x<0$, let $n = \infty,F(-1) = \frac{1}{(1-q)^2}$

Let $v(0,S) = F(-1)$, we can obtain the formulation for this question:

$v(x, S)=\left\{\begin{array}{ll}v(0,S)= \frac{1}{(1-q)^2} & \text { if } x=0 \\ p x+q v(x-1, S) & \text { if } 1 \leq x \leq S \\ v(S, S) & \text { if } S<x\end{array}\right.$

Consequently, $v(S,S) = p \sum_{i=0}^S q^i(S-i) + q^S\frac{1}{(1-q)^2} = \\ \quad S - \frac{q(1-q^S)}{p} + q^S \frac{1}{(1-q)^2}$.

\section*{4-5}
Define $~G^n (x) = -x + \sum_i^m f^{n-1}(x+r_i)p_i$.

The derivative of $G^n (x)$ equals $(-1 + \sum_i^m f^{'n-1}(x+r_i)p_i)$, because $f^{n-1}(x+r_i)$ is a linear function of $x$, and the coefficient of $x$ is 1. So $\sum_i^m f^{'n-1}(x+r_i)p_i = \sum_i^m p_i = 1-p < 1$. So the derivative of $G^n (x)$ is less than 0. Therefore, $G^n (x) < G^n (x+1)$. According to Exercise 3.4, the decision rule we decided is indeed optimal.

\end{document}
